\newcommand\hmmax{0} % ok
\newcommand\bmmax{0} % ok
%---------------------------------------------------------
% packages
%---------------------------------------------------------
\usepackage[utf8]{inputenc} % ok
\usepackage[T1]{fontenc} % ok
\usepackage[english]{babel} % ok
\usepackage{csquotes} % ok
\usepackage{amsmath} % ok
\usepackage{amsfonts} % ok
\usepackage{amsbsy} % ok
\usepackage{amssymb} % ok
\usepackage{bm} % ok
\usepackage{blindtext} % ok
\usepackage[nice]{nicefrac} % ok
% \usepackage{media9}
\usepackage{array} % ok
\usepackage{float} % ok
\usepackage{pifont} % ok
\usepackage{color} % ok
\usepackage{fancybox} % ok
% \usepackage{animate}
\usepackage{graphicx} % ok
\usepackage[absolute,overlay]{textpos} % ok
\usepackage{times} % ok
\usepackage{setspace} % ok
\usepackage{subcaption} % ok
\usepackage{mathtools} % ok

\usepackage{changepage} % ok
\usepackage{tikz} % ok
\usepackage{hyperref} % ok
\usepackage{breakurl} % ok
\usepackage{etoolbox} % ok
\usepackage[restart]{parnotes} % ok
\usepackage[style=authoryear,giveninits=true]{biblatex} % ok

% \usepackage{fontawesome}
\usepackage{fontawesome5} % ok

\usepackage[stretch=10]{microtype} % ok

\usepackage{multimedia} % ok

\usepackage{minibox} % ok

\usepackage[most]{tcolorbox} % ok  colored and framed text boxes

\usepackage{booktabs} % ok

\usepackage{layouts} % ok

\usepackage[yyyymmdd]{datetime} % ok

\usepackage{nicematrix} % ok

\usepackage{cancel} % ok

% \usepackage{MnSymbol} % removed during session 3
\usepackage{mathdots} % ok

\usepackage[percent]{overpic} % ok

%---------------------------------------------------------
% colors
%---------------------------------------------------------
\definecolor{csugold}{HTML}{C8C372}
\definecolor{csugold}{HTML}{C8C372}
\definecolor{niceblue}{HTML}{4D79A7}
\definecolor{nicered}{HTML}{E15656}
\definecolor{nicegreen}{HTML}{6DC350}
\definecolor{nicegray}{HTML}{F2F2F2}
\definecolor{xmediumgray}{HTML}{AAAAAA}
\definecolor{codegreen}{HTML}{487F07}
\definecolor{nicepink}{HTML}{EF8EDB}
\definecolor{darkergray}{HTML}{818181}


\newcommand{\cdigits}[1]{\textcolor{niceblue}{#1}}

% \usepackage{enumitem}

%---------------------------------------------------------
% commands
%---------------------------------------------------------

% all math is display style instead of inline
% \everymath{\displaystyle}

% add bib file
\addbibresource{../../../references.bib}

\newlength\sectionsep
\setlength\sectionsep{6pt}% default 1.875ex plus 1fill

\captionsetup{aboveskip=3pt}
\captionsetup{belowskip=0pt}


% boxes
\newtcolorbox{xgreen}[1]{%
    boxrule = 1pt,
    boxsep = 2pt,
    breakable,
    colframe = black,
    enhanced jigsaw,
    interior hidden,
    parbox = false,
    shadow={2mm}{-1mm}{0mm}{black!50!white},
    title = #1,
    fonttitle = \bfseries,
    title style={left color=xprimarycolor!85!black,right color=xprimarycolor},
}

\newtcolorbox{xgeneralbox}[3]{%
    boxrule = 1pt,
    boxsep = 3pt,
    breakable,
    colframe = edgecolor,
    enhanced jigsaw,
    interior hidden,
    parbox = false,
    title = #1,
    coltitle = black!85,
    title style={left color=#2, right color=#3},
    fontupper={\small\color{textcolor}},
    fontlower={\small},
    top = 0pt,
    bottom = 0pt,
    left = 2pt,
    right = 2pt,
}
%     fonttitle = \bfseries,

% REMOVED!
% \newenvironment{myremark}
%     {\begin{xgeneralbox}{\scalebox{0.75}{\faComment~\textbf{Remark}}}{xgraycolor}{xgraycolor}}
%     {\end{xgeneralbox}}

\newenvironment{myreview}[1]
    {\begin{xgeneralbox}{\scalebox{0.75}{\faComment~\textbf{Review (Session #1)}}}{xgraycolor}{xgraycolor}}
    {\end{xgeneralbox}}

% \newenvironment{myfuture}
%     {\begin{xgeneralbox}{\scalebox{0.75}{\faFastForward~\textbf{Looking Ahead}}}{niceblue!25!white}{niceblue!25!white}}
%     {\end{xgeneralbox}}

\newenvironment{myhint}
    {\begin{xgeneralbox}{\scalebox{0.75}{\faLightbulbO~\textbf{Hint}}}{xgraycolor}{xgraycolor}}
    {\end{xgeneralbox}}

% \newenvironment{myquestion}
%     {\begin{xgeneralbox}{\scalebox{0.75}{\faQuestionCircle~\textbf{Question}}}{nicered!25!white}{nicered!25!white}}
%     {\end{xgeneralbox}}

% \newenvironment{myimportant}
%     {\begin{xgeneralbox}{\scalebox{0.75}{\faExclamationCircle~\textbf{Important Remark}}}{nicegreen!25!white}{nicegreen!25!white}}
%     {\end{xgeneralbox}}

\newtcolorbox{xempty}[1]{%
    boxrule = 1pt,
    boxsep = 2pt,
    % breakable,
    colframe = black,
    enhanced jigsaw,
    interior hidden,
    parbox = false,
    bottom = -7pt,
    title = #1,
    fonttitle = \bfseries,
    coltitle = black!75,
    title style={left color=xgraycolor!85!black,right color=xgraycolor},
}

\newtcolorbox{xxgreen}[1]{%
    boxrule = 1pt,
    boxsep = 1pt,
    left = 3pt,
    right = 3pt,
    breakable,
    colframe = black,
    enhanced jigsaw,
    parbox = false,
    shadow={2mm}{-1mm}{0mm}{black!50!white},
    title = {#1},
    fonttitle = \bfseries,
    title style={left color=xprimarycolor, right color=xprimarycolor},
    colback=tcolorboxbg,
}
%     interior hidden,
% title = {\centering {\hspace*{-3pt}{\parbox[][11pt][c]{0pt}{}}{#1}}},
    % title = {\\ \centering },

\newtcolorbox{xxgreencompact}{%
    boxrule = 1pt,
    boxsep = 0pt,
    left = 0pt,
    right = 0pt,
    breakable,
    colframe = black,
    enhanced jigsaw,
    interior hidden,
    parbox = false,
    shadow={2mm}{-1mm}{0mm}{black!50!white},
    % title = #1,
    % fonttitle = \bfseries,
    % title style={left color=xprimarycolor!85!black,right color=xprimarycolor},
}

\newtcolorbox{theorembox}[1]{%
    boxrule = 1pt,
    boxsep = 2pt,
    breakable,
    colframe = black,
    enhanced jigsaw,
    interior hidden,
    parbox = false,
    title = #1,
    fonttitle = \bfseries,
    coltitle = black!0,
    title style={left color=xprimarycolor!85!black, right color=xprimarycolor!80!white},
    fontupper=\small,
    fontlower=\small,
    top = 0pt,
    bottom = 0pt,
    left = 3pt,
    right = 3pt,
}


\usepackage{ragged2e} % ok
\RaggedRight

\makeatletter
\def\@minipagerestore{\RaggedRight}
\makeatother

\newcommand{\mylinewhite}[1]{\textcolor{white}{\rule{#1}{0.8pt}}}

\newsavebox{\picbox}
\newcommand{\cutpic}[3]{
  \savebox{\picbox}{\includegraphics[width=#2]{#3}}
  \tikz\node [draw, rounded corners=#1, line width=0.5pt,
    color=black, minimum width=\wd\picbox,
    minimum height=\ht\picbox, path picture={
      \node at (path picture bounding box.center) {
        \usebox{\picbox}};
    }] {};}

\renewenvironment{quote}[1][1em]
  {\itshape\begin{adjustwidth}{#1}{#1}}
  {\end{adjustwidth}}

\usepackage{accents} % ok
\newcommand{\xub}[1]{\underaccent{\bar}{#1}}
\newcommand{\xob}[1]{\bar{#1}}

\DeclareMathOperator*{\argmax}{arg\,max}
\DeclareMathOperator*{\argmin}{arg\,min}
\DeclarePairedDelimiter\abs{\lvert}{\rvert}
\DeclarePairedDelimiter\norm{\lVert}{\rVert}
\DeclarePairedDelimiter\bracket{[}{]}
\DeclarePairedDelimiter\paren{(}{)}
\DeclarePairedDelimiter\curl{\lbrace}{\rbrace}
\DeclarePairedDelimiter\ceil{\lceil}{\rceil}
\DeclarePairedDelimiter\floor{\lfloor}{\rfloor}

\newcommand{\xcolor}[1]{\textsl{\textsf{#1}}}

\newcommand{\pmin}{\underaccent{\bar}{\bm{P}}}
\newcommand{\pmax}{\bar{\bm{P}}}

\newcommand{\rmin}{\underaccent{\bar}{\bm{R}}}
\newcommand{\rmax}{\bar{\bm{R}}}

% \input{.config/AlgorithmStyle}
% \import{../}{.config/AlgorithmStyle}

% \input{.config/vAlgorithm}
% \import{../}{.config/vAlgorithm}

\usepackage{amsmath} % ok
\usepackage{amsfonts} % ok
\usepackage{amssymb} % ok
\usepackage{setspace} % ok
\usepackage{textcomp} % ok
\usepackage{multicol} % ok

\newcounter{NumberInTable}
\newcommand{\LTNUM}{\stepcounter{NumberInTable}{(\theNumberInTable)}}

\newcommand{\Laplace}[1]{\ensuremath{\mathcal{L}{\{#1\}}}}
\newcommand{\InvLap}[1]{\ensuremath{\mathcal{L}^{-1}{\left[#1\right]}}}

\newcommand{\xoptional}[1]{\textcolor{csugold}{#1}}
\newcommand{\myoptional}{\xoptional{[Optional]}}

\newcommand{\xtheorem}[1]{{[\textbf{\textcolor{xprimarycolor}{#1}}]}}

\newcommand{\xeqspace}[1]{\quad \text{({#1})}}

\newcommand{\xeq}[1]{\text{\textcolor{textcoloremph}{#1}}}

\newcommand{\xeqq}[1]{\textcolor{textcoloremph}{#1}}

\newcommand{\xtableau}[1]{\textcolor{niceblue}{#1}}


\setcounter{MaxMatrixCols}{20}

\newcommand{\pd}{\phantom{-}}

\usepackage[normalem]{ulem} % ok



% \addtobeamertemplate{block begin}{%
%     \setlength{\textwidth}{0.9\textwidth}
% }{}


\addtobeamertemplate{block begin}{%
    \centering%
    \setlength{\textwidth}{0.85\textwidth}%
}{}

\addtobeamertemplate{block alerted begin}{%
    \centering%
    \setlength{\textwidth}{0.85\textwidth}%
}{}

\addtobeamertemplate{block example begin}{%
    \centering%
    \setlength{\textwidth}{0.85\textwidth}%
}{}


\BeforeBeginEnvironment{block}{\begin{adjustbox}{minipage={\linewidth}, center}}
\AfterEndEnvironment{block}{\end{adjustbox}\vspace*{0.05in}}

\usepackage{adjustbox} % ok

\usepackage{multirow} % ok

\newtheorem{xalgorithm}{Algorithm}
\newtheorem{xdefinition}{Definition}

\newcommand{\ubar}[1]{\underaccent{\bar}{#1}}
\newcommand{\obar}[1]{\bar{#1}}


\usepackage{listings} % ok

\lstset{numbers=left, numberstyle=\tiny, stepnumber=1, firstnumber=1,
    numbersep=5pt,
    language=Matlab,
    stringstyle=\ttfamily,
    basicstyle=\ttfamily,
    showstringspaces=false,
    commentstyle=\color{nicegreen},
    keywordstyle=\color{textcolor},
}
%     commentstyle=\color{codegreen},

\lstset{upquote=true}

\usepackage{contour} % ok

\renewcommand{\ULdepth}{1.8pt}
\contourlength{0.8pt}

\newcommand{\myuline}[1]{%
  \uline{\phantom{#1}}%
  \llap{\contour{white}{#1}}%
}

\DeclareCiteCommand{\citetitle}
  {\boolfalse{citetracker}%
   \boolfalse{pagetracker}%
   \usebibmacro{prenote}}
  {\ifciteindex
     {\indexfield{indextitle}}
     {}%
   \printtext[bibhyperref]{\printfield[citetitle]{labeltitle}}}
  {\multicitedelim}
  {\usebibmacro{postnote}}

% \newcommand{\reviewmaterials}{Please see the course \href{\courseurl{/files/24090855}}{Linear Algebra and Calculus Review} for resources and examples}

\newcommand{\sessiontaskremark}{%
\begin{myimportant}

\begin{itemize}

\item \xhy{Submit a document (.pdf {\small\faFilePdf{}} or .doc/x {\small\faFileWord}) summarizing your answers to the questions}

\item Submit all code used to answer the questions

\item A .zip {\small\faFileArchive{}} file containing these files is preferred if you have multiple files

\item Please provide guidance on how I should review your submission if there are many files

\end{itemize}

\end{myimportant}%
}%

\newcommand\twodigits[1]{%
   \ifnum#1<10 0#1\else #1\fi
}


\newtcolorbox{xfigbox}{%
    boxrule = 1pt,
    boxsep = 1pt,
    left = 3pt,
    right = 3pt,
    breakable,
    colframe = edgecolor,
    enhanced jigsaw,
    parbox = false,
    shadow={2mm}{-1mm}{0mm}{black!50!white},
    colback=tcolorboxbg,
}
%     title style={left color=xprimarycolor, right color=xprimarycolor},
%     fonttitle = \bfseries,
%     title = {#1},
%     interior hidden,
% title = {\centering {\hspace*{-3pt}{\parbox[][11pt][c]{0pt}{}}{#1}}},
    % title = {\\ \centering },

\newlength{\xfigwidth}
\newcommand{\myfigbox}[2]{%
\settowidth{\xfigwidth}{\includegraphics[scale=#2]{#1}}
\addtolength{\xfigwidth}{20pt}
\begin{columns}
\begin{column}{\xfigwidth}
\begin{xfigbox}
\centering
\includegraphics[scale=#2]{#1}
\end{xfigbox}
\end{column}
\end{columns}%
}

\newcommand{\myfigboxmin}[2]{%
\settowidth{\xfigwidth}{\includegraphics[scale=#2]{#1}}
\addtolength{\xfigwidth}{20pt}
\begin{column}{\xfigwidth}
\begin{xfigbox}
\centering
\includegraphics[scale=#2]{#1}
\end{xfigbox}
\end{column}%
}

\newcommand{\myoverpic}[3]{%
\settowidth{\xfigwidth}{\includegraphics[scale=#2]{#1}}
\addtolength{\xfigwidth}{20pt}
\begin{columns}
\begin{column}{\xfigwidth}
\begin{xfigbox}
\centering
% \includegraphics[scale=#2]{#1}
\begin{overpic}[scale=#2]{#1}
#3
\end{overpic}
\end{xfigbox}
\end{column}
\end{columns}%
}

\newcommand{\myoverpicmin}[3]{%
\settowidth{\xfigwidth}{\includegraphics[scale=#2]{#1}}
\addtolength{\xfigwidth}{20pt}
\begin{column}{\xfigwidth}
\begin{xfigbox}
\centering
% \includegraphics[scale=#2]{#1}
\begin{overpic}[scale=#2]{#1}
#3
\end{overpic}
\end{xfigbox}
\end{column}%
}

\newcommand{\xversion}{\textcolor{textcoloremph}{Version \today{\ }{@}\currenttime}}

\newcommand{\myversion}{%
\begin{textblock}{6}(11.6455,15.5325)
\scriptsize\xversion
\end{textblock}%
}

\def\hy#1{\parbox{\linewidth}{#1}} % helper for using command several times

\newcommand{\matlabfunctionmin}[2]{%
\def \matlabfunctiontext {\detokenize{#1}}%
\def \matlabfunctionurl {#2}%
\texttt{\href{\matlabfunctionurl}{\matlabfunctiontext}}%
}%

\newcommand{\matlabfunction}[2]{%
\def \matlabfunctiontext {\detokenize{#1}}
\def \matlabfunctionurl {#2}
\xmatlabfunction{\href{\matlabfunctionurl}{\matlabfunctiontext}}
}

\newcommand{\xmatlabfunction}[1]{%
\begin{columns}%
\begin{column}{0.08\textwidth}\centering
\vspace*{0.04in}
\if\darkmode1
\includegraphics[width=\textwidth]{../../../.config/figures/matlab-3-dark.pdf}
\else
\includegraphics[width=\textwidth]{../../../.config/figures/matlab-3.pdf}
\fi
\small
\texttt{#1}
\end{column}%
\end{columns}%
}

% \newcommand*{\eqcolorboxed}[1]{%
% \colorbox{xprimarycolor!05!white}{$\displaystyle{#1}$}
% }%

\newcommand*{\eqcolorboxed}[1]{%
\colorbox{bgcolorlight}{$\displaystyle{#1}$}
}%

% Syntax: \colorboxed[<color model>]{<color specification>}{<math formula>}
\newcommand*{\colorboxed}{}
\def\colorboxed#1#{%
  \colorboxedAux{#1}%
}
\newcommand*{\colorboxedAux}[3]{%
  % #1: optional argument for color model
  % #2: color specification
  % #3: formula
  \begingroup
    \colorlet{cb@saved}{.}%
    \color#1{#2}%
    \boxed{%
      \color{cb@saved}%
      #3%
    }%
  \endgroup
}

\NiceMatrixOptions{
code-for-first-row = \color{textcoloremph},
code-for-last-row = \color{textcoloremph},
code-for-first-col = \color{textcoloremph},
code-for-last-col = \color{textcoloremph}
}

\makeatother
\let\OldNabla\nabla
\RenewDocumentCommand{\nabla}{e_}{%
    \OldNabla
    \IfValueT{#1}{%
        _{\!#1\,}
    }%
}
\makeatletter

\newcommand{\myurlhref}[1]{\textcolor{textcoloremph}{{\footnotesize\faLink}\,#1}}
\newcommand{\myurl}[1]{\textcolor{textcoloremph}{{\footnotesize\faLink}\,\url{#1}}}
\newcommand{\myhref}[2]{\textcolor{textcoloremph}{{\footnotesize\faLink}\,\href{#1}{#2}}}
% \newcommand{\myvideourl}[1]{\textcolor{textcoloremph}{\faVideo~\url{#1}}}

\newcommand{\myurlmin}[1]{\textcolor{textcoloremph}{{\footnotesize\faLink}\,\href{https://\detokenize{#1}}{\detokenize{#1}}}}


\newcommand{\myvideourl}[1]{\textcolor{textcoloremph}{{\footnotesize\faVideo}\,\href{https://\detokenize{#1}}{\detokenize{#1}}}}

\newcommand{\xneed}[1]{\textcolor{red}{(#1)}}

\newcommand{\myparnotespace}{\hfill\null}

\newcommand{\fulllineparnote}[1]{\parnote{\makebox[\textwidth][l]{#1}}}

\newcommand{\ec}[1]{\xeqq{\faStar{} Extra credit ({#1} pts):}}

% \newcommand*{\tran}{^{\mkern-1.5mu\mathsf{T}}}
% \newcommand*{\tran}{^\top}
\newcommand*{\tran}{^{\intercal}}



\newcommand{\tv}[1]{\textcolor{nicered}{#1}}

%------------------
% semester-specific commands
\newcommand{\courseid}{170091}
\newcommand{\courseurl}[1]{https://colostate.instructure.com/courses/\courseid#1}

\newcommand{\reviewmaterialsurl}{\href{\courseurl{/files/28134160}}{\textcolor{textcoloremph}{{\footnotesize\faLink}~Linear Algebra and Calculus Review}}}

\newcommand{\reviewmaterials}{Please see the course \reviewmaterialsurl{} for resources and examples}




% \usepackage{letltxmacro}
% \usepackage{pgffor}


% %% https://tex.stackexchange.com/questions/14393/how-keep-a-running-list-of-strings-and-then-process-them-one-at-a-time
% \newcommand\FigList{}
% \newcommand\AddFigToList[1]{\edef\FigList{\FigList#1,}}

% \LetLtxMacro{\OldIncludegraphics}{\includegraphics}
% \renewcommand{\includegraphics}[2][]{%
%     \AddFigToList{#2}%
%     \OldIncludegraphics[#1]{#2}%
% }

% \newcommand*{\ShowListOfFigures}{%
%     \typeout{Figures included were}%
%     \foreach \x in \FigList {%
%         \par\x% <-- uncomment if you want the list in the PDF as well
%         \typeout{ \x}
%     }%
% }
% \AtEndDocument{\ShowListOfFigures}

% \newcommand{\listtermsname}{Terms}
% \newlistof{listofterms}{lot}{\listtermsname}

% \usepackage{tocloft}


\usepackage{xparse} % ok

\ExplSyntaxOn

\NewDocumentCommand{\createlist}{m}
 {
  \seq_new:c { g_mo_list_#1_seq }
 }
\NewDocumentCommand{\additem}{mmm}
 {
  \seq_gput_right:cn { g_mo_list_#1_seq } { \__mo_list_do:nn {#2}{#3} }
 }
\NewDocumentCommand{\makelist}{O{}m}
 {
  \group_begin:
  \keys_set:nn { mo/list } { #1 }
  \l__mo_list_pre_tl
  \seq_use:cn { g_mo_list_#2_seq } { }
  \l__mo_list_post_tl
  \group_end:
 }
\keys_define:nn { mo/list }
 {
  pre     .tl_set:N  = \l__mo_list_pre_tl,
  post    .tl_set:N  = \l__mo_list_post_tl,
  command .code:n    = \cs_set:Nn \__mo_list_do:nn { #1 },
  pre     .initial:n = { \begin{itemize}\itemsep=3pt },
  post    .initial:n = { \end{itemize} },
  command .initial:n = \item[{\small\faHashtag}] {#1} {#2},
 }
\ExplSyntaxOff

% %
% \newcommand{\pagetarget}[2]{%
%   \phantomsection%
%   \label{#1}%
%   \hypertarget{#1}{#2}%
% }

\makeatletter
\newcommand\needlabel[1]{%
  \@ifundefined{r@#1}{%
    \label{#1}%
  }{%
    % \ref{KN:#1}%
  }%
}
\makeatother

%
% \newcommand{\pagetarget}[2]{%
%   \phantomsection%
%   \label{#1}%
%   \hypertarget{#1}{#2}%
% }
\makeatletter
\newcommand{\pagetarget}[2]{%
  \phantomsection%
  % \@ifundefined{r@#1}{%
    \label{#1}%
  % }{%
   \hypertarget{#1}{#2}%
  % }%
}
\makeatother


\newcommand{\xhiformat}[1]{\textit{\textcolor{textcoloremph}{#1}}}

\newcommand{\xhiname}[1]{\textcolor{textcoloremph}{\small\faHashtag}\xhiformat{#1}}

% https://tex.stackexchange.com/questions/252566/calculate-the-hash-md5-or-otherwise-of-a-string
% \newcommand{\xhilabel}[1]{#1}
\newcommand{\xhilabel}[1]{\pdfmdfivesum{#1}}

% \newcommand{\xhi}[1]{\additem{listone}{\hyperlink{#1}{\xhiformat{#1}}}{\text{ is on page }\pageref{#1}}\pagetarget{#1}{\xhiformat{#1}}}
% \newcommand{\xhi}[1]{\additem{listone}{\hyperlink{\xhilabel{#1}}{\xhiformat{#1}}}{\text{ is on Slide }\ref{\xhilabel{#1}}}\pagetarget{\xhilabel{#1}}{\xhiname{#1}}}
\newcommand{\xhi}[1]{\additem{listone}{\hyperlink{\xhilabel{#1}}{\xhiformat{#1}}}{\text{ is on Slide }\ref{\xhilabel{#1}}}\pagetarget{\xhilabel{#1}}{\xhiname{#1}}}


\createlist{listone} % A description list specifically

\makeatletter
\g@addto@macro{\UrlBreaks}{\UrlOrds}
\makeatother

\newcommand{\meqcounter}{\addtocounter{equation}{-1}}


\renewcommand{\qedsymbol}{\textcolor{textcoloremph}{\openbox}}

% \providecommand{\udots}{\Udots}
% \makeatletter
% \DeclareRobustCommand{\Udots}{%
%   \vcenter{\offinterlineskip
%     \halign{%
%       \hbox to .8em{##}\cr
%       \hfil.\cr\noalign{\kern.2ex}
%       \hfil.\hfil\cr\noalign{\kern.2ex}
%       .\hfil\cr}%
%   }%
% }
% \makeatother

\def\udots#1{\cdot^{\cdot^{\cdot^{#1}}}}


% push something to the bottom of the slide
\newcommand{\btVFill}{\vskip0pt plus 1filll}


% itemize hyphenation command
\newcommand{\xhy}[1]{\parbox[t]{\linewidth}{\strut#1\strut}\par
}


%---------------------------------------------------------
% ip stuff
%---------------------------------------------------------
% https://www.privacypolicies.com/blog/sample-copyright-notice/

\newcommand{\xcopyright}{\textcolor{textcoloremph}{\textcopyright{} \the\year{} Daniel R. Herber. All Rights Reserved.}}

\newcommand{\mycopyright}{%
\begin{textblock}{6}(0.6455,15.5325)
\scriptsize\xcopyright
\end{textblock}%
}

%---------------------------------------------------------
% final slides
%---------------------------------------------------------

% term slides
\newcommand{\termslides}{%
\miniframesoff%
\bgroup%
\section*{\texorpdfstring{\faHashtag}{Terms}}%
\begin{frame}[c,allowframebreaks]{\xtitlemark Terms}%
\label{sec:terms}%
\makelist{listone}

\end{frame}%
\egroup%
\miniframeson%
}

% reference slides
\newcommand{\refslides}{%
\miniframesoff%
\bgroup%
\section*{\texorpdfstring{\scalebox{0.87485779}{\faBookmark}}{References}}%
\begin{frame}[allowframebreaks]{\xtitlemark References}%
\label{sec:references}%
% \nocite{*}
\printbibliography[heading=none]
\end{frame}%
\egroup%
\miniframeson%
}


%---------------------------------------------------------
% customized bookmarks in the PDF
%---------------------------------------------------------
% https://tex.stackexchange.com/questions/74696/global-bookmark-settings
% \hypersetup{
%   bookmarksnumbered=true
% }
\usepackage{bookmark}[2019/12/03] % ok
\bookmarksetup{
  open,
  openlevel=3,
  addtohook={%
    \ifnum\bookmarkget{level}=2%
      \bookmarksetup{bold}%
    \fi
  },
}


\usepackage{setspace} % ok


%---------------------------------------------------------
\newenvironment{myslide}[2]
{\begin{frame}[#1]%
\frametitle{{\xtitlemark #2}}%
}%
{\parnotes\end{frame}}
% \newenvironment{myslide}[2]
% {\begin{frame}[#1]{\xtitlemark #2}%
% \frametitle{\xtitlemark #2}%
% }%
% {\parnotes\end{frame}}

%---------------------------------------------------------
\newenvironment{myslidefragile}[2]
{\begin{frame}[fragile,environment=myslidefragile,#1]%
\frametitle{{\xtitlemark {#2}}}%
}%
{\parnotes\end{frame}}

%---------------------------------------------------------
\newenvironment{mytheorem}[1]
{\setcounter{theorem}{\insertframenumber-1}
\begin{theorem}[#1]
}%
{\end{theorem}}

%---------------------------------------------------------
\newenvironment{mydefinition}[1]
{\setcounter{xdefinition}{\insertframenumber-1}
\begin{xdefinition}[#1]
}%
{\end{xdefinition}}

%---------------------------------------------------------
\newenvironment{myalgorithm}[1]
{\setcounter{xalgorithm}{\insertframenumber-1}
\begin{xalgorithm}[#1]
}%
{\end{xalgorithm}}

%---------------------------------------------------------
\newcommand{\myfaEdit}{{\small\faEdit}\,}








%---------------------------------------------------------
% \usepackage{awesomebox}
% \setlength{\aweboxvskip}{0pt}

% % \newenvironment{myremarknew}
% % {\begin{awesomeblock}[black!85][\abLongLine]{2pt}{\faComment}{xgraycolor}\small
% % }%
% % {\end{awesomeblock}}

% \usepackage{tabularx}
% \usepackage{colortbl}
% \usepackage{makecell}

% \newcolumntype{z}{>{\small\columncolor{xgraycolor}}X}
% \newcolumntype{?}{!{{\vrule width 2pt}}}

% \newenvironment{myremarknew}{\bgroup%
% \tabularx{\linewidth}{p{0.08\linewidth}!{\color{black!85}\vrule width 2pt}z}
% \begin{minipage}{0.08\linewidth}\color{black!85}\small\centering Remark \\ \faComment\end{minipage} &}
% {\endtabularx\egroup}

% \newenvironment{myquestionnew}{\bgroup%
% \tabularx{\linewidth}{p{0.08\linewidth}!{\color{nicered}\vrule width 2pt}z}%
% \begin{minipage}{0.08\linewidth}\color{nicered}\small\centering Question \\ \faQuestionCircle\end{minipage} & }
% {\endtabularx\egroup}

% % \newenvironment{myfuturenew}{\bgroup%
% % \tabularx{\linewidth}{p{1.1cm}!{\color{niceblue}\vrule width 2pt}z}%
% % \vspace*{-\parskip}%
% % \fcolorbox{red}{gray}{\begin{minipage}{1.1cm}\color{niceblue}\small\centering Looking \\ Ahead \\ \faFastForward\end{minipage}} & }
% % {\endtabularx\egroup}

% % \newenvironment{myfuturenew}{\bgroup%
% % \tabularx{\linewidth}{p{0.08\linewidth}!{\color{niceblue}\vrule width 2pt}z}%
% % \vspace{-0.8\baselineskip} \textcolor{niceblue}{\centering\footnotesize Looking  Ahead \faFastForward} & }
% % {\endtabularx\egroup}

% % \newenvironment{myfuturenew}{\bgroup%
% % \tabularx{\linewidth}{p{0.08\linewidth}!{\color{niceblue}\vrule width 2pt}z}%
% % \scalebox{0.8}{%
% % \begin{tikzpicture}
% % \node[draw=none, align=center, text width=0.08\linewidth, text=niceblue,] at (0,0) {Looking  Ahead \faFastForward};
% % \end{tikzpicture}} & }
% % {\endtabularx\egroup}


% \newenvironment{myfuturenew}{\bgroup%
% \tabularx{\linewidth}{p{0.08\linewidth}!{\color{niceblue}\vrule width 2pt}z}%
% \makecell{\textcolor{niceblue}{\small Looking\\  Ahead\\ \faFastForward}} & }
% {\endtabularx\egroup}



% % % \textcolor{black!85}{\textbf{\footnotesize Remark}}

% % \newtcolorbox{xremarknew}{%
% %   sidebyside,
% %   sidebyside align=top,
% %   lefthand width=1.1cm,
% %   sidebyside gap=14pt,
% %   lower separated=true,
% %   enhanced,
% %   bicolor,
% %   fontupper={\color{black!85}},
% %   fontlower={\small\color{textcolor}},
% %   sharp corners,
% %   segmentation style = {solid, line width = 2pt},
% %   top = 2pt,
% %   bottom = 2pt,
% %   left=0pt,
% %   right=4pt,
% %   colback=xgraycolor,
% %   colbacklower=green!10,
% %   frame hidden,
% % }
% % %   blanker,
% % %   frame hidden,
% %   % colback=white,
% % %   borderline vertical={1pt}{0pt}{black},

% % % leftrule=18mm,
% % %   segmentation style={draw=black,solid},
% %   % colframe=yellow!80!black,
% % %   % boxrule=4pt,
% % %   
% % %   arc=0pt,
% % % width=\textwidth,

% % \newenvironment{myremarknew}
% %     {\begin{xremarknew} \centering {\footnotesize\textbf{Remark}} \\ \faComment \tcblower}
% %     {\end{xremarknew}}




\newtcolorbox{xawesomebox}[3]{%
    sidebyside,
    sidebyside adapt=left,
    sidebyside align=center seam,
    bicolor,
    overlay={\draw[solid,#1,line width=2pt] (segmentation.north)--(segmentation.south);},
    center,
    % segmentation style={solid,line width=2pt},
    % boxrule=0pt,
    % arc=0pt,
    sidebyside gap=3.5mm,
    left=1mm,
    right=1mm,
    top=1.5mm,
    bottom=1.5mm,
    boxsep=0mm,
    fontupper={\small\color{#1}}, % niceblue!25!white
    fontlower={\small\color{textcolor}},
    frame hidden,
    colback=bgcolor,
    colbacklower=bgcolorlight,
}



\usepackage{environ}% ok http://ctan.org/pkg/environ

\newcommand{\impcmd}[5]{\begin{adjustwidth}{-9mm}{0mm}
\tcbsidebyside[bicolor,
sidebyside adapt=left,
sidebyside align=center seam,
overlay={\draw[solid,#4,line width=2pt] (segmentation.north)--(segmentation.south);},
center,
% boxrule=0pt,
arc=0pt,
sidebyside gap=3.5mm,
left=1mm,
right=1mm,
top=1.5mm,
bottom=1.5mm,
boxsep=0mm,
fontupper={\small\color{#4}}, % niceblue!25!white
fontlower={\small\color{textcolor}},
frame hidden,
colback=bgcolor,
colbacklower=bgcolorlight,
before skip=0.25\baselineskip,
after skip=0.35\baselineskip,
]{%
{\begin{minipage}{\widthof{#1}}\centering \small
#2\\\vspace{1mm} \scalebox{1.25}{#3}
\end{minipage}}
}{%
#5
}
\end{adjustwidth}}

\NewEnviron{myimportant}{%
\impcmd{Important}{Important}{\faExclamationCircle}{nicegreen}{\BODY}}

\NewEnviron{myremark}{%
\impcmd{Remark}{Remark}{\faComment}{niceblue}{\BODY}}

\NewEnviron{myfuture}{%
\impcmd{Looking}{Looking\\ Ahead}{\faFastForward}{nicepink}{\BODY}}

\NewEnviron{myquestion}{%
\impcmd{Question}{Question}{\faQuestionCircle}{nicegreen}{\BODY}}

\NewEnviron{myoptionalbox}{%
\impcmd{Optional}{Optional}{\faComment}{niceblue}{\BODY}}


\newcommand{\myiconstyle}[1]{\textcolor{xprimarycolor!75}{\scalebox{5}{#1}}}

\newcommand{\mysummaryicon}{\textcolor{xprimarycolor!75}{\scalebox{5}{\faLightbulb[regular]}}}

\newcommand{\mytaskicon}{\textcolor{xprimarycolor!75}{\scalebox{5}{\faEdit[regular]}}}

\newcommand{\myreadingicon}{\textcolor{xprimarycolor!75}{\scalebox{5}{\faBook}}}

\newcommand{\myreferenceicon}{\textcolor{xprimarycolor!75}{\scalebox{5}{\faBookmark}}}

\newcommand{\myoutlineicon}{\textcolor{xprimarycolor!75}{\scalebox{5}{\faListOl}}}

% \begin{adjustwidth}{-9mm}{0mm}
% \tcbsidebyside[bicolor,
% sidebyside adapt=left,
% sidebyside align=center seam,
% overlay={\draw[solid,nicegreen,line width=2pt] (segmentation.north)--(segmentation.south);},
% center,
% % segmentation style={solid,line width=2pt},
% % boxrule=0pt,
% % arc=0pt,
% sidebyside gap=3.5mm,
% left=1mm,
% right=1mm,
% top=1.5mm,
% bottom=1.5mm,
% boxsep=0mm,
% fontupper={\small\color{nicegreen}}, % niceblue!25!white
% fontlower={\small\color{textcolor}},
% frame hidden,
% colback=bgcolor,
% colbacklower=bgcolorlight,
% ]{%
% {\begin{minipage}{\widthof{Important}}\centering \small
% Important\\ Remark\\\vspace{1mm} \scalebox{1.25}{\faExclamationCircle}
% \end{minipage}}
% }{%
% Interior points are only ``interior'' to the inequality constraints. If equality constraints are present, any feasible point will satisfy them. Since it is not possible to be interior to an equality constraint, some authors use the term \xhi{relative interior points}.
% }
% \end{adjustwidth}

